\documentclass{resume}

\newcommand{\en}[1]{#1}
\newcommand{\zh}[1]{}

\zh{\usepackage{xeCJK}}
% \zh{\setCJKmainfont{SourceHanSansHWSC-Regular}}
% \zh{\setCJKsansfont{SourceHanSansHWSC-Regular}}
% \zh{\setCJKmonofont{SourceHanSansHWSC-Regular}}

\begin{document}

\name{\en{Yiwei Yang}\zh{杨易为}}
\basicInfo{
      \email{victoryang00@ucsc.edu} \textperiodcentered\
      \homepage[asplos.dev]{https://asplos.dev}
      \github[victoryang00]{https://github.com/victoryang00}
}


\section{\en{Education}\zh{教育经历}}
\en{\datedsubsection{\textbf{UC Santa Cruz}, Ph.D. Student}{08/2022 -- 06/2028}}
\zh{\datedsubsection{\textbf{加州大学圣克鲁兹分校}, 在读Ph.D.}{2018/09 -- 2022/06}}
\begin{itemize}
      \item \en{Major: Computer Science, advised by \href{https://arquinn.github.io/}{Andrew Quinn}. TA of Computer Architecture}
            \zh{计算机科学,导师为\href{https://arquinn.github.io/}{Andrew Quinn},2022体系结构助教}
\end{itemize}
\en{\datedsubsection{\textbf{ShanghaiTech University}, Undergraduate}{09/2018 -- 06/2022}}
\zh{\datedsubsection{\textbf{上海科技大学}, 本科}{2018/09 -- 2022/06}}
\begin{itemize}
      \item \en{Major: Computer Science, finished Compiler, Network, Database, OS, CA, Convex, RL, Parallel Computing. TA of Compiler}
            \zh{计算机科学,完成编译原理,操作系统,网络,数据库,并行计算,凸优化,强化学习等课程,2021/2编译原理助教}
\end{itemize}


\section{\en{Work Experience}\zh{工作经历}}
\en{\datedsubsection{\textbf{\href{https://www.jumptrading.com/}{Jump Trading}}, Shanghai, China}{07/2020 -- 09/2020}}
\zh{\datedsubsection{\textbf{\href{https://www.jumptrading.com/}{跃申有限公司}}}{2020/07 -- 2020/08}}
\en{\role{Linux Team}{Production Engineer Intern}}
\zh{\role{Linux 组}{HPC自动化运维实习}}
\begin{itemize}
      \item \en{High Frequency Trade Order Book simulation applying Linear.Regression Method.}
            \zh{应用线性回归法进行高频交易盘账模拟。}
      \item \en{Applied salt and jinja to automate scheduling of jobs and assigning affinity of cpu cores in Linux DevOps.}
            \zh{应用salt和jinja在Linux DevOps中自动调度作业和分配cpu核心绑定。}
      \item \en{Applied gobidng of gobpf to try IOVisor stuff.}
            \zh{应用eBPF的gobinding进行io侦测。}
\end{itemize}
\section{\en{Research Experience}\zh{研究经历}}

\en{\datedsubsection{\textbf{\href{https://www.ssrc.ucsc.edu/index.html}{Storage Systems Research Center}, UC Santa Cruz}}{08/2022 -- Present}}
\zh{\datedsubsection{\textbf{\href{https://www.ssrc.ucsc.edu/index.html}{加州大学圣克鲁兹分校存储中心}}}{08/2022 -- 现在}}
\en{\role{Graduate Research}{Assistant}}
\zh{\role{研究生科研}{助理}}
\begin{itemize}
      \item \small{ \en{Understand the performance characteristics of CXL.mem systems. Data-driven far memory allocation, prefetching, and replacement policies. } \zh{理解CXL.mem的性能分析。数据驱动的远端内存分配、预取和替换策略。}}
      \item{\small \en{Simulate CXL.mem for data center applications, Emulate CXL.cache for reliable driver and kernel fuzzing.}
      \zh{模拟CXL.mem数据中心应用,为kernel可靠性模拟CXL.cache设备。}}
      \\
      \textbf{Yiwei Yang} Pooneh Safayenikoo, Jiacheng Ma, Tanvir Ahmad Khan, Andrew Quinn. \textbf{"CXLMemSim: A pure software simulated CXL.mem for performance characterization."} Yarch23  .
      \item \small{ \en{Make Hardware Software Co-design for on CXL.cache data movement}
      \zh{设计一种硬件软件协同的CXL.cache数据迁移方法。}}
      \item \small{ \en{Make Virtual Machine migration based on WebAssembly and WASI}
      \zh{设计一种基于WebAssembly和WASI的虚拟机迁移方法。}}
\end{itemize}

\en{\datedsubsection{\textbf{\href{http://mir.cs.illinois.edu/marinov/}{Darko Marinov's Lab}, \href{https://illinois.edu}{University of Illinois at Urbana-Champaign}}}{07/2021 -- 10/2021}}
\zh{\datedsubsection{\textbf{\href{https://illinois.edu}{伊利诺伊大学厄巴纳-香槟分校}\href{http://mir.cs.illinois.edu/marinov/}{Darko Marinov组}}}{2021/07 -- 2021/10}}
\en{\role{Research Experience for Undergraduate}{Remote Program}}
\zh{\role{本科生科研}{暑研项目}}
\begin{itemize}
      \item \en{Investigating \href{https://victoryang00.cn/wordpress/2022/08/09/final-report-of-program-analysis/}{Order dependent Flaky Tests} by Dynamic Taint Analysis and fix the underlying concurrent bugs.}
            \zh{使用动态污点分析在Flaky数据集上进行\href{https://victoryang00.cn/wordpress/2022/08/09/final-report-of-program-analysis/}{数据依赖变量检测},并修复java写的并行系统中的并发bug。}
\end{itemize}

\en{\datedsubsection{\textbf{\href{http://s3l.shanghaitech.edu.cn/}{System} \href{https://toast-lab.gitee.io/}{Lab}, ShanghaiTech University}}{07/2019 -- 06/2020}}
\zh{\datedsubsection{\textbf{\href{http://s3l.shanghaitech.edu.cn/}{上海科技大学系统}\href{https://toast-lab.gitee.io/}{实验室}}}{2019/07 -- 2021/06}}
\en{\role{Undergraduate Research}{Intern}}
\zh{\role{本科生科研}{实习}}
\begin{itemize}
      \item \small{ \en{A2D uses the cost of attacking an input for robustness evaluation and identifies those less robust examples as adversarial since less robust examples are easier to attack.}
      \zh{A2D使用攻击输入的成本来进行鲁棒性评估,并将那些鲁棒性较差的label diff rate这一指标guide生成对抗样本。}}\\
      Zhao, Zhe, Guangke Chen, Jingyi Wang, \textbf{Yiwei Yang}, Fu Song, and Jun Sun. \textbf{"Attack as Defense: Characterizing Adversarial Examples using Robustness."} ISSTA21  .
      \item \en{Researching \href{https://github.com/LEAFERx/movable}{MOVE language} in Diem currency source code to protect against Arithmetic Overflow, Timestamp Difference.}
            \zh{研究Diem源码中的\href{https://github.com/LEAFERx/movable}{MOVE language},提高抵御时间戳攻击、整数溢出等的安全性。}
      \item \en{Understanding the \href{https://github.com/victoryang00/pmemable}{real access mechanism} of Memory Mode Optane Memory and XPBuffer by reverse-engineering methods.}
            \zh{通过逆向工程的相关手段去理解在memory mode 下傲腾内存以及XPBuffer的\href{https://github.com/victoryang00/pmemable}{内存置换算法逻辑}。}
\end{itemize}


% \section{\en{Portfolios}\zh{团队项目}}
% \datedsubsection{\href{https://github.com/chocopy-llvm/chocopy-llvm}{\textbf{ChocoPy Compiler}}}{\tiny{\en{A compiler to LLVM IR and to riscv assembly of python 3.6 subset as ShanghaiTech Compiler class project}
% \zh{一个从 python3.6 子集编译至 LLVM IR 和 riscv 汇编的编译器,作为编译原理课的 project}
% }}

% \datedsubsection{\href{https://github.com/LemonHX/ioring-rs}{\textbf{IORing Rust}}}{\tiny{\en{IORing Rust for windows, future support \href{https://github.com/bytedance/monoio}{monoio}}\zh{windows上的IORing Rust版,未来支持\href{https://github.com/bytedance/monoio}{monoio}}}}

\section{\en{Skills}\zh{技能}}
\begin{itemize}[parsep=0.25ex]
      \item \en{\textbf{Programming Languages}:
                  not limited to any specific language,
                  and experienced in Python/C++/Rust,
                  comfortable with Golang/C/Java/Scala/TypeScript (in random order).}
            \zh{\textbf{编程语言}:
                  不局限于特定编程语言,且尤其熟悉 Python/C++/Rust 等,
                  了解 Golang/C/Java/Scala/TypeScript 等。}

      \item \en{\textbf{System}:
                  Specialist in Compiler \& Performance Analysis, familiar with \texttt{LLVM}, \texttt{Linux perf}, \texttt{eBPF}}
            \zh{\textbf{系统}:
                  熟悉各种编译器及操作系统内核的概念与设计,熟悉各种内核性能调优工具,例如\texttt{LLVM}, \texttt{Linux perf}, \texttt{eBPF}。}

      \item \en{\textbf{Machine Learning}:
                  familiar with general knowledge of machine \& reinforce learning.}
            \zh{\textbf{机器学习}:
                  熟悉经典机器与强化学习算法。}
\end{itemize}

\section{\en{Miscellaneous}\zh{其他}}
\begin{itemize}
      \item \en{Interests: Computer Architecture, Storage System, Formal Methods, etc.}
            \zh{兴趣:高性能计算、体系结构、存储、形式化验证安全等。}
      \item \en{Awards:}
      \zh{所获奖项:}
      \begin{itemize}
      \item \en{Lead \href{https://hpc.geekpie.club}{GeekPie\_HPC} \textit{Ranked 2}, SC-SCC21. \textit{Ranked 4}, ISC22. Advise Not-Slow-Slug \textit{Ranked 2}, ISC23.}
            \zh{带领\href{https://hpc.geekpie.club}{GeekPie\_HPC} SCC21 团队第2名,ISC22 团队第4名,指导 Not-Slow-Slug 团队 ISC23 团队第2名。}
      \item \en{As a member of \href{https://github.com/0x238e/Vchain}{0x238e} \textit{Best award}, Bitrun, Hang Zhou, 2019.}
            \zh{带领\href{https://github.com/0x238e/Vchain}{0x238e} Bitrun Hackathon 比赛最佳奖}
      \item \en{\textit{Second Award}, Shanghai CTF invitation competition, 2019.}
            \zh{上海2019年 CTF 邀请赛比赛二等奖}
      \end{itemize}
\end{itemize}

\end{document}
